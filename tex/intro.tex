\section{Introduction}\label{sec:intro}

Learning to generate realistic time series remains a fundamental challenge for machine learning systems.
Deep Neural Networks (DNNs) have shown great promise at modeling complex datasets by learning a hierarchy of abstractions within data.
 \cite{bengio_representation_2013, bengio_neural_2003}
Specifically, Convolutional Neural Networks (CNNs) learn convolution filters at gradually increasing scales, extending to high-level structural representations \cite{krizhevsky_imagenet_2012, lecun_convolutional_1998}.
As a result, DNNs are now commonplace for non-linear modeling relationships in both sequential and non-sequential data
\cite{goodfellow_deep_2016, maaloe_auxiliary_nodate, oord_wavenet_2016, thomsen_deepfret_2020}.

When modeling data, we optimize the model's likelihood $p(\dot)$ over the data $\mathbf{x}$.
For sequence modeling, we calculate the likelihood of each step of the data $x_t\in\mathbf{x}$ conditioned on all earlier sequence steps, i.e.
\begin{equation}\label{eq:llh}
    p(\mathbf{x}) = \prod^N_{t=1}  p(x_t | x_{<t}) , \mathbf{x}\in \mathbb{R}^N
\end{equation}

The most commonly used models for sequential data are based on  Recurrent Neural Network (RNN) units, more specifically, Long Short-Term Memory (LSTM) and Gated Recurrent Units (GRU) \cite{hochreiter_long_1997, karim_lstm_2018, sundermeyer_lstm_2012, cho_learning_2014}.
Recurrent architectures condition the likelihood in \cref{eq:llh} on earlier timesteps through one or more "hidden states" $h_t$ for each timestep $x_t$, as shown in \cref{fig:intro-rnn}, i.e.
\begin{equation}\label{eq:llh-rnn}
    p(x_t, h_t | x_{<t}) = p(x_t | x_{t-1}, h_{t-1})
\end{equation}

This structure allows architectures with recurrent units to model arbitrarily long sequences with unbounded context \cite{dauphin_language_2017}.
However, the same recurrence causes a series of problems in training neural networks. 
Recurrent architectures depend on backpropagation through time (BPTT) to accurately calculate gradients. 
This type of backpropagation involves calculating gradients over the entire input sequence for each weight update. 
For a sequence of length $N$, the derivative of the weights with respect to the $i$th element in the sequence is calculated $N-i$ times, which results in an $\mathcal{O}(n^2)$ complexity for a single example \cite{williams_gradient-based_1995}.
Apart from being computationally expensive, this approach can render model training unstable, as small perturbations in the input can have a "butterfly effect" on the gradient computation - this is commonly known as vanishing and exploding gradients \cite{cuellar_application_2006}.

% This backpropagation slows down training and leaves the gradient updates in recurrent architectures susceptible to variations in input sample order. 
% As a result, training with backpropagation through time is unstable, as vanishing and exploding gradients commonly occur in recurrent architectures. \cite{cuellar_application_2006}



Alternative to recurrent architectures, convolutional autoregressive architectures model the likelihood in \cref{eq:llh} by limiting the input sequence to a "receptive field" (RF) of size $R$.
This is shown in \cref{fig:intro-ar}.
For convolutional autoregressive models, the hidden state $h_t$ disappears from \cref{eq:llh-rnn}, and we can express the likelihood of a single timestep as a function of the element and the preceding $R$ elements of the sequence: 
\begin{equation}
    p(x_t | x_{\geq(t-R-1), <t}) = f(x_t, x_{\geq(t-R-1), <t})
\end{equation}\label{eq:llh-ar-rf}

This renders the likelihood of the data, \cref{eq:llh}, as a product of these outputs, where the likelihood of each timestep can be computed independently of one another:
\begin{equation}
    p(\mathbf{x}) = \prod^N_{t=R+1}  p(x_t | x_{\geq(t-R-1), <t}) = \prod^N_{t=R+1} f(x_t, x_{\geq(t-R-1), <t})
\end{equation}\label{eq:llh-ar-rf-long}


The "stateless" property of convolutional autoregressive models speeds up training, as the model can train on all steps of the input sequence in parallel \cite{salimans_pixelcnn_2017}.
However, this property limits convolutional autoregressive models compared to RNNs; autoregressive models do not natively model data dependencies longer than their receptive field.
\footnote{
When training RNNs with truncated backpropagation through time - the truncation effectively enforces a "receptive field" of the input sequence for the sake of stability and faster convergence
\cite{miller_stable_2019, cuellar_application_2006}.
}


\begin{figure}[t]  
\centering 
  \begin{subfigure}[b]{0.45\linewidth}
  \resizebox{\columnwidth}{!}
    {
    \begin{tikzpicture}%[transform canvas={scale=0.5}]
    % Input nodes
    \node[circle, 
        minimum size = 10mm,
        draw,
        % fill=orange!30
    ] (Input-0) at (0,0) {$x_{t}$};
    
    \foreach \i in {1,2,3}
    {
        \node[circle, 
            minimum size = 10mm,
            draw,
            % fill=orange!30,
            ] (Input-\i) at (-2*\i,0) {$x_{t-\i}$};
    }
    \node[circle, minimum size = 10mm,% fill=orange!30,
            ] (Input-ldots) at (-2*4,0) {\dots};
    \node[circle, minimum size = 10mm,% fill=orange!30,
            above=of Input-ldots] (RNN-ldots) {\dots};
    
    % Draw RNNs
    \foreach \i in {0,1,2,3}
    {
    \node[rectangle, draw, above=of Input-\i, minimum height=1cm, minimum width=1cm] (RNN-\i) {RNN};
    }

    % Draw Outputs

    \node[circle, minimum size = 10mm, draw, fill=gray!30, above=of RNN-0] (Output-0) {$\hat{x}_{t+1}$};
    \node[circle, minimum size = 10mm, draw, fill=gray!30, above=of RNN-1] (Output-1) {$\hat{x}_{t}$};    
    \node[circle, minimum size = 10mm, draw, fill=gray!30, above=of RNN-2] (Output-2) {$\hat{x}_{t-1}$};
    \node[circle, minimum size = 10mm, draw, fill=gray!30, above=of RNN-3] (Output-3) {$\hat{x}_{t-2}$};
    % draw arrows
    \foreach \i in {0,1,2,3}
    {
    \draw[->] (Input-\i.north) -- (RNN-\i.south);
    \draw[->] (RNN-\i.north) -- (Output-\i.south);
    }

    \draw[->,dashed] (RNN-ldots.east) -- (RNN-3.west) ;
    \draw[->] (RNN-3.east) -- (RNN-2.west) node[midway, above] {$h_{t-3}$};
    \draw[->] (RNN-2.east) -- (RNN-1.west) node[midway, above] {$h_{t-2}$};
    \draw[->] (RNN-1.east) -- (RNN-0.west) node[midway, above] {$h_{t-1}$};
    \draw[->] (RNN-0.east) -- +(20pt,0) node[midway, above] {$h_{t}$};
    
    % % Arrow from Hat to next 
    % \draw[->] (Output-3.east) edge[bend right=15] (Input-2.west);
    % \draw[->] (Output-2.east) edge[bend right=15] (Input-1.west);
    % \draw[->] (Output-1.east) edge[bend right=15] (Input-0.west);

    \draw[->, dashed] (Output-1.east) 
        to[out=-30, in=150]
        % edge[bend left=30] 
    (Input-0.west) node[left] {$\mathcal{L}_{t}$}   ;

\end{tikzpicture}%%
    }
  \caption{Recurrent Architecture. The recurrent unit can be either a Recurrent Neural Network, a Long Short-Term Memory cell, or a Gated Recurrent Unit.
  } \label{fig:intro-rnn}  
  \end{subfigure}
  \hfill
  \begin{subfigure}[b]{0.45\linewidth}
  \resizebox{\columnwidth}{!}
    {
    \begin{tikzpicture}
    % Input nodes
    \node[circle, 
    minimum size = 10mm,
    draw,
    % fill=orange!30
    ] (Input-0) at (0,0) {$x_{t}$};
    
    \foreach \i in {1,2,3}
    {
        \node[circle, 
            minimum size = 10mm,
            draw,
            % fill=orange!30,
            ] (Input-\i) at (-1.25*\i,0) {$x_{t-\i}$};
    }
    
    % Network
    \node[rectangle, draw, above=of Input-1, minimum height=1cm, minimum width=1cm] (AR-Model) {AR model};
    
    % dots node
    \node[circle, 
        minimum size = 10mm,
        % draw,
        % fill=orange!30,
        ] (Input-dots) at (-4.87,0) {$\dots$};
    % R node at pos 5 
    \node[circle, 
            minimum size = 10mm,
            draw,
            % fill=orange!30,
            ] (Input-R) at (-6,0) {$x_{t-R}$};
            
    \node[circle, 
        minimum size = 10mm,
        % draw,
        % fill=orange!30,
        ] (Input-R-1) at (-7,0) {\dots};
    
    % \draw[->] (Input-R.north) -- (AR-Model.south);
    
    % Arrows
    \foreach \i in {1,2,3,R}
    {
        \draw[->] (Input-\i.north) -- (AR-Model.south);
    }

    % Output
    \node[circle,
     minimum size = 10mm,
    fill=gray!30,
    draw,
    above=of AR-Model
    ] (Output-0) {$\hat{x}_{t}$};

    \draw[->] (AR-Model.north) -- (Output-0.south);

    % Arrow from Hat to next 
    \draw[->, dashed] (Output-0.east) edge[bend left=30] node[midway, right] {$\mathcal{L}_{t}$} (Input-0.north) ;

\end{tikzpicture}%%
    }
    \caption{
    Convolutional Autoregressive Architecture.
    The illustrated model can be any feed-forward neural network architecture.  
    } \label{fig:intro-ar}  
  \end{subfigure}
  \caption{
  Comparison between Recurrent and Autoregressive architectures for computing $\hat{x}_t$. Note that to estimate $\hat{x}_t$ accurately, the RNN needs to calculate its output multiple times, while the autoregressive model only needs to calculate its output once. 
  }
\end{figure}  

% FROM HERE ON, WaveNet introduction
The WaveNet is a convolutional autoregressive model for generating high-fidelity audio, using dilated causal convolutions to maximize the receptive field and reduce training cost \cite{oord_wavenet_2016, kalchbrenner_neural_2017}.
% The WaveNet uses dilated causal convolutions with the last input timestep masked to enforce next-step prediction.
Dilated convolution layers inflate the kernel size by a dilation rate $d$ by inserting $d-1$ "holes" between kernel elements
\cite{dutilleux_implementation_1990}.
WaveNets stack dilated convolution layers with exponentially increasing dilation factors, ensuring that every time step is seen once by the network, as shown in \cref{fig:intro-wavenet}.
Dilated convolutional layers allow WaveNets to have exponentially growing receptive fields, so stacks of dilated convolution appear as a more lightweight alternative to regular convolutional setups with similar receptive fields
\cite{yu_multi-scale_2016}.
A more detailed comparison of the number of weights is found in \cref{appx:dilation-weights}.
Reusing weights across the input also prevents gradient collapse backward in time, as the same weights are reused across the receptive field.
% TODO: move to model section?

WaveNet's intended use is a component of larger Text-To-Speech (TTS) systems, as a strong autoregressive decoder from predicted spectrograms to high-fidelity audio. 
\cite{wang_tacotron_2017, oord_parallel_2017, oord_neural_2018, chorowski_unsupervised_2019}
Notably, the VQ-VAE uses conditioned WaveNets for decoding audio from latent representations and has successfully generated adjacent phonemes and word fragments. \cite{oord_neural_2018, garbacea_low_2019, dieleman_variable-rate_2021}
Likewise, modern production systems for TTS like Tacotron 2 use WaveNet to decode sequences of Mel spectrograms and similar audio representations in place of the Griffin-Lim audio reconstruction algorithm \cite{shen_natural_2018}.
On the other hand, autoregressive models have received criticism for excelling at capturing local signal structure while missing long-range correlations. \cite{dieleman_challenge_2018}
The original WaveNet paper discusses the same problems and attributes the low power in modeling long-range correlations to the limited receptive field of their architecture. \cite{oord_wavenet_2016, shen_natural_2018}
This presents a problem, as the receptive field of the WaveNet architecture is about 300 ms, significantly less than a sequence of words, much less a complete sentence.\cite{oord_wavenet_2016}
An increased receptive field is unavoidable for a deterministic autoregressive model like WaveNet to model longer-range dependencies.


For a convolutional autoregressive model to generate realistic-sounding speech, it must contain some form of language modeling capability.
Stacked convolutional layers learn arbitrarily large contexts with multiple hierarchies of abstraction, given a sufficiently large input size \cite{lecun_convolutional_1998}. 
As language contains a hierarchical structure, phonemes making up words, which build phrases and construct sentences, we expect a convolutional model like WaveNet to model language well.
Other convolutional autoregressive models have been reported to rival LSTM-based architectures for language modeling \cite{bai_empirical_2018}.
Notably, Dauphin and collaborators used stacks of gated convolutional units to calculate the language context from a "patch" of $k$ preceding words \cite{dauphin_language_2017}.
As convolutional autoregressive architectures model both audio and text well, we hypothesize that a WaveNet model can produce coherent speech learned directly from audio if the receptive field is large enough.


\paragraph{In this work} we investigate whether a WaveNet architecture learns higher-level semantic information within an audio sample. 
We investigate this by probing the following hypotheses:
\begin{enumerate}
    \item\label{asmpt:rf-first} WaveNet's receptive field is the main limiting factor for modeling long-range dependencies. Increasing the receptive field will increase the predictive power of the model.
    \item\label{asmpt:represent} WaveNet's stacked convolutional layers learn good representations of speech, fully or partially extracting semantic features in audio.
    \item\label{asmpt:lang-mod} WaveNet's hierarchical structure makes it suitable to learn priors over representations of speech such as text. 
    \item\label{asmpt:end2end} Finally, if hypotheses \ref{asmpt:represent} and \ref{asmpt:lang-mod} find support, a sufficiently large WaveNet architecture trained on speech can generate coherent words and sentence fragments.
\end{enumerate}


\begin{figure}
    \centering
    \resizebox{0.7\columnwidth}{!}{
    \begin{tikzpicture}
\node[circle,draw,fill=blue!50,minimum height=5mm,] (Input-15) at (15,0) {};
\node[circle,draw,fill=gray!30,minimum height=5mm] (Hidden1-15) at (15,2) {};
\node[circle,draw,fill=gray!30,minimum height=5mm] (Hidden2-15) at (15,4) {};
\node[circle,draw,fill=gray!30,minimum height=5mm] (Hidden3-15) at (15,6) {};
\node[circle,draw,fill=yellow!50,minimum height=5mm] (Output-15) at (15,8) {};

\node[right=of Input-15, align=left] {
\textbf{Input}
};
\node[right=of Hidden1-15, align=left] {
\textbf{Hidden1}\\
Dilation = 1
};
\node[right=of Hidden2-15, align=left] {
\textbf{Hidden2}\\
Dilation = 2
};
\node[right=of Hidden3-15, align=left] {
\textbf{Hidden3}\\
Dilation = 4
};
\node[right=of Output-15, align=left] {
\textbf{Output}\\
Dilation = 8
};


\draw[->, line width=1.5] (15,0.3) -- (15, 1.7);
\draw[->, line width=1.5] (15,2.3) -- (15, 3.7);
\draw[->, line width=1.5] (15,4.3) -- (15, 5.7);
\draw[->, line width=1.5] (15,6.3) -- (15, 7.7);


\foreach \x in {0, 1, ..., 14}{
	\ifthenelse{\intcalcMod{\x}{2}=0}{
		\draw[->, line width=1.5] (\x+0.15,0.3) -- (\x+0.85, 1.7);
	}{
		\draw[->, line width=1.5] (\x,0.3) -- (\x, 1.7);
	}
	\ifthenelse{\intcalcMod{\x}{4}=1}{
		\draw[->, line width=1.5] (\x+0.3,2.3) -- (\x+1.7, 3.7);
	}{}
	\ifthenelse{\intcalcMod{\x}{4}=3}{
		\draw[->, line width=1.5] (\x,2.3) -- (\x, 3.7);
	}{}
	\ifthenelse{\intcalcMod{\x}{8}=3}{
		\draw[->, line width=1.5] (\x+0.3,4.3) -- (\x+3.7, 5.7);
	}{}
	\ifthenelse{\intcalcMod{\x}{8}=7}{
		\draw[->, line width=1.5] (\x,4.3) -- (\x, 5.7);
	}{}
	\ifthenelse{\intcalcMod{\x}{16}=7}{
		\draw[->, line width=1.5] (\x+0.3,6.3) -- (\x+7.7, 7.7);
	}{}
	\ifthenelse{\intcalcMod{\x}{16}=15}{
		\draw[->, line width=1.5] (\x,6.3) -- (\x, 7.7);
	}{}
}

\foreach \x in {0, 1, ..., 14}{
	\draw[fill=blue!50] (\x,0) circle [radius=0.25];
	\draw[fill=gray!30] (\x, 2) circle [radius=0.25];
	\draw[fill=gray!30] (\x, 4) circle [radius=0.25];
	\draw[fill=gray!30] (\x, 6) circle [radius=0.25];
	\draw[fill=yellow!30] (\x, 8) circle [radius=0.25];
}


% caption on the right
% \node[right=of Input-15] {Hello};


\end{tikzpicture}%}
    \caption{
    Illustration of a 4-layer WaveNet architecture with exponentially increasing dilation $d_i=2^i, i\in [0, 3]$ and kernel size 2.
    This results in a receptive field of size $2^4=16$. 
    }
    \label{fig:intro-wavenet}
\end{figure}

% In addition, we investigate the tradeoff between the number of residual channels and the number of dilation cycles in the WaveNet architecture in terms of performance. 
% Finally, we provide an analysis of the performance limits of the WaveNet when using a categorical output distribution for predicting multiple timesteps in the future. 

% \paragraph{In this work} we present the following contributions:
% \begin{itemize}
%     \item We demonstrate that WaveNet's chief performance limit is the number of residual channels in its architecture, rather than it's gross receptive field.
%     \item We report comparable likelihood scores between WaveNet and other generative sequence models.
%     \item We analyze WaveNet's performance using both categorical and DMoL output distributions.
% \end{itemize}




