% Options for packages loaded elsewhere
\PassOptionsToPackage{unicode}{hyperref}
\PassOptionsToPackage{hyphens}{url}
%
\documentclass[
  ignorenonframetext,
]{beamer}
\usepackage{pgfpages}
\setbeamertemplate{caption}[numbered]
\setbeamertemplate{caption label separator}{: }
\setbeamercolor{caption name}{fg=normal text.fg}
\beamertemplatenavigationsymbolsempty
% Prevent slide breaks in the middle of a paragraph
\widowpenalties 1 10000
\raggedbottom
\setbeamertemplate{part page}{
  \centering
  \begin{beamercolorbox}[sep=16pt,center]{part title}
    \usebeamerfont{part title}\insertpart\par
  \end{beamercolorbox}
}
\setbeamertemplate{section page}{
  \centering
  \begin{beamercolorbox}[sep=12pt,center]{part title}
    \usebeamerfont{section title}\insertsection\par
  \end{beamercolorbox}
}
\setbeamertemplate{subsection page}{
  \centering
  \begin{beamercolorbox}[sep=8pt,center]{part title}
    \usebeamerfont{subsection title}\insertsubsection\par
  \end{beamercolorbox}
}
\AtBeginPart{
  \frame{\partpage}
}
\AtBeginSection{
  \ifbibliography
  \else
    \frame{\sectionpage}
  \fi
}
\AtBeginSubsection{
  \frame{\subsectionpage}
}
\usepackage{lmodern}
\usepackage{amssymb,amsmath}
\usepackage{ifxetex,ifluatex}
\ifnum 0\ifxetex 1\fi\ifluatex 1\fi=0 % if pdftex
  \usepackage[T1]{fontenc}
  \usepackage[utf8]{inputenc}
  \usepackage{textcomp} % provide euro and other symbols
\else % if luatex or xetex
  \usepackage{unicode-math}
  \defaultfontfeatures{Scale=MatchLowercase}
  \defaultfontfeatures[\rmfamily]{Ligatures=TeX,Scale=1}
\fi
\usetheme[]{Metropolis}
% Use upquote if available, for straight quotes in verbatim environments
\IfFileExists{upquote.sty}{\usepackage{upquote}}{}
\IfFileExists{microtype.sty}{% use microtype if available
  \usepackage[]{microtype}
  \UseMicrotypeSet[protrusion]{basicmath} % disable protrusion for tt fonts
}{}
\makeatletter
\@ifundefined{KOMAClassName}{% if non-KOMA class
  \IfFileExists{parskip.sty}{%
    \usepackage{parskip}
  }{% else
    \setlength{\parindent}{0pt}
    \setlength{\parskip}{6pt plus 2pt minus 1pt}}
}{% if KOMA class
  \KOMAoptions{parskip=half}}
\makeatother
\usepackage{xcolor}
\IfFileExists{xurl.sty}{\usepackage{xurl}}{} % add URL line breaks if available
\IfFileExists{bookmark.sty}{\usepackage{bookmark}}{\usepackage{hyperref}}
\hypersetup{
  pdftitle={Generative Modelling of Sequential Data},
  pdfauthor={Magnus Sletfjerding},
  hidelinks,
  pdfcreator={LaTeX via pandoc}}
\urlstyle{same} % disable monospaced font for URLs
\newif\ifbibliography
\setlength{\emergencystretch}{3em} % prevent overfull lines
\providecommand{\tightlist}{%
  \setlength{\itemsep}{0pt}\setlength{\parskip}{0pt}}
\setcounter{secnumdepth}{-\maxdimen} % remove section numbering

\title{Generative Modelling of Sequential Data}
\subtitle{M.Sc. thesis in collaboration with Corti ApS}
\author{Magnus Sletfjerding}
\date{February 9th, 2022}

\begin{document}
\frame{\titlepage}

\begin{frame}{Contents}
\protect\hypertarget{contents}{}
\tableofcontents
\end{frame}

\hypertarget{introduction}{%
\section{Introduction}\label{introduction}}

\begin{frame}{Why is it interesting to study hierarchies of information
in time series?}
\protect\hypertarget{why-is-it-interesting-to-study-hierarchies-of-information-in-time-series}{}
\begin{itemize}
\item
  Most data we work with has some hierarchical structure

  \begin{itemize}
  \tightlist
  \item
    Text
  \item
    Video
  \item
  \end{itemize}
\item
  Human brains process hierarchies of information natively

  \begin{itemize}
  \tightlist
  \item
    Human-like AI requires hierarchical processing
  \end{itemize}
\item
  All real-world data has a time dimension.
\end{itemize}
\end{frame}

\begin{frame}{Sequence modeling}
\protect\hypertarget{sequence-modeling}{}
Sequence modeling optimize the model's likelihood \(p(\cdot)\) over the
data \(\mathbf{x}\), by conditioning the probability of \(x_t\) on
previous timesteps: \[
p(\mathbf{x}) = \prod^N_{t=1}  p(x_t | x_{<t}) , \hspace{1cm}  \mathbf{x}\in \mathbb{R}^N
\]
\end{frame}

\begin{frame}{Recurrent vs.~Convolutional Autoregressive models}
\protect\hypertarget{recurrent-vs.-convolutional-autoregressive-models}{}
\begin{figure}[t]  
\centering 
  \begin{subfigure}[b]{0.45\linewidth}
  \resizebox{\columnwidth}{!}
    {
    \begin{tikzpicture}%[transform canvas={scale=0.5}]
    % Input nodes
    \node[circle, 
        minimum size = 10mm,
        draw,
        % fill=orange!30
    ] (Input-0) at (0,0) {$x_{t}$};
    
    \foreach \i in {1,2,3}
    {
        \node[circle, 
            minimum size = 10mm,
            draw,
            % fill=orange!30,
            ] (Input-\i) at (-2*\i,0) {$x_{t-\i}$};
    }
    \node[circle, minimum size = 10mm,% fill=orange!30,
            ] (Input-ldots) at (-2*4,0) {\dots};
    \node[circle, minimum size = 10mm,% fill=orange!30,
            above=of Input-ldots] (RNN-ldots) {\dots};
    
    % Draw RNNs
    \foreach \i in {0,1,2,3}
    {
    \node[rectangle, draw, above=of Input-\i, minimum height=1cm, minimum width=1cm] (RNN-\i) {RNN};
    }

    % Draw Outputs

    \node[circle, minimum size = 10mm, draw, fill=gray!30, above=of RNN-0] (Output-0) {$\hat{x}_{t+1}$};
    \node[circle, minimum size = 10mm, draw, fill=gray!30, above=of RNN-1] (Output-1) {$\hat{x}_{t}$};    
    \node[circle, minimum size = 10mm, draw, fill=gray!30, above=of RNN-2] (Output-2) {$\hat{x}_{t-1}$};
    \node[circle, minimum size = 10mm, draw, fill=gray!30, above=of RNN-3] (Output-3) {$\hat{x}_{t-2}$};
    % draw arrows
    \foreach \i in {0,1,2,3}
    {
    \draw[->] (Input-\i.north) -- (RNN-\i.south);
    \draw[->] (RNN-\i.north) -- (Output-\i.south);
    }

    \draw[->,dashed] (RNN-ldots.east) -- (RNN-3.west) ;
    \draw[->] (RNN-3.east) -- (RNN-2.west) node[midway, above] {$h_{t-3}$};
    \draw[->] (RNN-2.east) -- (RNN-1.west) node[midway, above] {$h_{t-2}$};
    \draw[->] (RNN-1.east) -- (RNN-0.west) node[midway, above] {$h_{t-1}$};
    \draw[->] (RNN-0.east) -- +(20pt,0) node[midway, above] {$h_{t}$};
    
    % % Arrow from Hat to next 
    % \draw[->] (Output-3.east) edge[bend right=15] (Input-2.west);
    % \draw[->] (Output-2.east) edge[bend right=15] (Input-1.west);
    % \draw[->] (Output-1.east) edge[bend right=15] (Input-0.west);

    \draw[->, dashed] (Output-1.east) 
        to[out=-30, in=150]
        % edge[bend left=30] 
    (Input-0.west) node[left] {$\mathcal{L}_{t}$}   ;

\end{tikzpicture}%%
    }
  %\caption{Recurrent Architecture. %The recurrent unit can be either a Recurrent Neural Network, a Long Short-Term Memory cell, or a Gated Recurrent Unit.
  %}  \label{fig:intro-rnn}  
  \end{subfigure}
  \hfill
  \begin{subfigure}[b]{0.45\linewidth}
  \resizebox{\columnwidth}{!}
    {
    \begin{tikzpicture}
    % Input nodes
    \node[circle, 
    minimum size = 10mm,
    draw,
    % fill=orange!30
    ] (Input-0) at (0,0) {$x_{t}$};
    
    \foreach \i in {1,2,3}
    {
        \node[circle, 
            minimum size = 10mm,
            draw,
            % fill=orange!30,
            ] (Input-\i) at (-1.25*\i,0) {$x_{t-\i}$};
    }
    
    % Network
    \node[rectangle, draw, above=of Input-1, minimum height=1cm, minimum width=1cm] (AR-Model) {AR model};
    
    % dots node
    \node[circle, 
        minimum size = 10mm,
        % draw,
        % fill=orange!30,
        ] (Input-dots) at (-4.87,0) {$\dots$};
    % R node at pos 5 
    \node[circle, 
            minimum size = 10mm,
            draw,
            % fill=orange!30,
            ] (Input-R) at (-6,0) {$x_{t-R}$};
            
    \node[circle, 
        minimum size = 10mm,
        % draw,
        % fill=orange!30,
        ] (Input-R-1) at (-7,0) {\dots};
    
    % \draw[->] (Input-R.north) -- (AR-Model.south);
    
    % Arrows
    \foreach \i in {1,2,3,R}
    {
        \draw[->] (Input-\i.north) -- (AR-Model.south);
    }

    % Output
    \node[circle,
     minimum size = 10mm,
    fill=gray!30,
    draw,
    above=of AR-Model
    ] (Output-0) {$\hat{x}_{t}$};

    \draw[->] (AR-Model.north) -- (Output-0.south);

    % Arrow from Hat to next 
    \draw[->, dashed] (Output-0.east) edge[bend left=30] node[midway, right] {$\mathcal{L}_{t}$} (Input-0.north) ;

\end{tikzpicture}%%
    }
   % \caption{
    %Convolutional Autoregressive Architecture.
    % The illustrated model can be any feed-forward neural network architecture.  
    %} \label{fig:intro-ar}  
  \end{subfigure}
  %\caption{
%  Comparison between Recurrent and Autoregressive architectures for computing $\hat{x}_t$. Note that to estimate $\hat{x}_t$ accurately, the RNN needs to calculate its output multiple times, while the autoregressive model only needs to calculate its output once. 
  %}
\end{figure}  

\begin{columns}[T]
\begin{column}{0.48\textwidth}
\begin{block}{Recurrent architectures}
\protect\hypertarget{recurrent-architectures}{}
Condition \(p(x_t|x_{<t})\) through one or more hidden states \(h_t\)
passed between timesteps: \[
p(x_t, h_t | x_{<t}) = p(x_t | x_{t-1}, h_{t-1})
\]
\end{block}
\end{column}

\begin{column}{0.48\textwidth}
\begin{block}{Autoregressive Architectures}
\protect\hypertarget{autoregressive-architectures}{}
Condition \(p(x_t|x_{<t})\) by viewing a receptive field of size \(R\)
of the input sequence. \[
p(\mathbf{x}) = \prod^N_{t=R+1} p(x_t | x_{\geq t-R+1, <t})
\]
\end{block}
\end{column}
\end{columns}
\end{frame}

\begin{frame}{WaveNet - Convolutional Autoregressive Sequence Modelling}
\protect\hypertarget{wavenet---convolutional-autoregressive-sequence-modelling}{}
\begin{figure}
    \centering
    \resizebox{0.7\columnwidth}{!}{
    \begin{tikzpicture}
\node[circle,draw,fill=blue!50,minimum height=5mm,] (Input-15) at (15,0) {};
\node[circle,draw,fill=gray!30,minimum height=5mm] (Hidden1-15) at (15,2) {};
\node[circle,draw,fill=gray!30,minimum height=5mm] (Hidden2-15) at (15,4) {};
\node[circle,draw,fill=gray!30,minimum height=5mm] (Hidden3-15) at (15,6) {};
\node[circle,draw,fill=yellow!50,minimum height=5mm] (Output-15) at (15,8) {};

\node[right=of Input-15, align=left] {
\textbf{Input}
};
\node[right=of Hidden1-15, align=left] {
\textbf{Hidden1}\\
Dilation = 1
};
\node[right=of Hidden2-15, align=left] {
\textbf{Hidden2}\\
Dilation = 2
};
\node[right=of Hidden3-15, align=left] {
\textbf{Hidden3}\\
Dilation = 4
};
\node[right=of Output-15, align=left] {
\textbf{Output}\\
Dilation = 8
};


\draw[->, line width=1.5] (15,0.3) -- (15, 1.7);
\draw[->, line width=1.5] (15,2.3) -- (15, 3.7);
\draw[->, line width=1.5] (15,4.3) -- (15, 5.7);
\draw[->, line width=1.5] (15,6.3) -- (15, 7.7);


\foreach \x in {0, 1, ..., 14}{
	\ifthenelse{\intcalcMod{\x}{2}=0}{
		\draw[->, line width=1.5] (\x+0.15,0.3) -- (\x+0.85, 1.7);
	}{
		\draw[->, line width=1.5] (\x,0.3) -- (\x, 1.7);
	}
	\ifthenelse{\intcalcMod{\x}{4}=1}{
		\draw[->, line width=1.5] (\x+0.3,2.3) -- (\x+1.7, 3.7);
	}{}
	\ifthenelse{\intcalcMod{\x}{4}=3}{
		\draw[->, line width=1.5] (\x,2.3) -- (\x, 3.7);
	}{}
	\ifthenelse{\intcalcMod{\x}{8}=3}{
		\draw[->, line width=1.5] (\x+0.3,4.3) -- (\x+3.7, 5.7);
	}{}
	\ifthenelse{\intcalcMod{\x}{8}=7}{
		\draw[->, line width=1.5] (\x,4.3) -- (\x, 5.7);
	}{}
	\ifthenelse{\intcalcMod{\x}{16}=7}{
		\draw[->, line width=1.5] (\x+0.3,6.3) -- (\x+7.7, 7.7);
	}{}
	\ifthenelse{\intcalcMod{\x}{16}=15}{
		\draw[->, line width=1.5] (\x,6.3) -- (\x, 7.7);
	}{}
}

\foreach \x in {0, 1, ..., 14}{
	\draw[fill=blue!50] (\x,0) circle [radius=0.25];
	\draw[fill=gray!30] (\x, 2) circle [radius=0.25];
	\draw[fill=gray!30] (\x, 4) circle [radius=0.25];
	\draw[fill=gray!30] (\x, 6) circle [radius=0.25];
	\draw[fill=yellow!30] (\x, 8) circle [radius=0.25];
}


% caption on the right
% \node[right=of Input-15] {Hello};


\end{tikzpicture}%}
    %\caption{
    %Illustration of a 4-layer WaveNet architecture with exponentially increasing dilation $d_i=2^i, i\in [0, 3]$ and kernel size 2.
    %This results in a receptive field of size $2^4=16$. 
    %}
    %\label{fig:intro-wavenet}
\end{figure}

\begin{itemize}
\tightlist
\item
  Common vocoder in Speech To Text production systems
\item
  No ``hidden state'' for representing earlier timesteps
\item
  Constrained to look back within receptive field
\end{itemize}
\end{frame}

\hypertarget{problem-and-hypotheses}{%
\section{Problem and Hypotheses}\label{problem-and-hypotheses}}

\begin{frame}{Main Problem with WaveNet \ldots{}}
\protect\hypertarget{main-problem-with-wavenet}{}
\begin{itemize}
\tightlist
\item
  excelling at capturing local signal structure
\item
  missing long-range correlations
\item
  low receptive field (300ms)
\item
  audio generation sounds like babbling
\end{itemize}
\end{frame}

\begin{frame}{Hypotheses Investigated}
\protect\hypertarget{hypotheses-investigated}{}
\begin{enumerate}
\tightlist
\item
  WaveNet's receptive field is the main limiting factor for modeling
  long-range dependencies.
\item
  WaveNet's stacked convolutional layers learn good representations of
  speech.
\item
  WaveNet's hierarchical structure makes it suitable to learn priors
  over representations of speech such as text.
\item
  A large WaveNet architecture trained on speech can generate coherent
  words and sentence fragments
\end{enumerate}
\end{frame}

\hypertarget{experiments}{%
\section{Experiments}\label{experiments}}

\begin{frame}{Experiment overview}
\protect\hypertarget{experiment-overview}{}
\end{frame}

\begin{frame}{Expanding Receptive Field By Stacking - Setup}
\protect\hypertarget{expanding-receptive-field-by-stacking---setup}{}
\begin{figure}
\centering
\resizebox{\columnwidth}{!}{
\begin{tikzpicture}
    \foreach \x in {16, 12, 8, 4} {
            \node[circle, draw, fill=black!20, minimum height=10mm,] (Stack-\intcalcDiv{\x}{4}-0) at (\x, 2.4) {};

            \node[circle, draw, fill=black!20, minimum height=10mm,] (Stack-\intcalcDiv{\x}{4}-1) at (\x, 2.6) {};

            \node[circle, draw, fill=black!20, minimum height=10mm,] (Stack-\intcalcDiv{\x}{4}-2) at (\x, 2.8) {};

            \node[circle, draw, fill=black!20, minimum height=10mm,] (Stack-\intcalcDiv{\x}{4}) at (\x, 3) {$x^*_{\intcalcDiv{\x}{4}}$};
        }


    \foreach \x in {16, 15, ..., 1}{
            \node[circle,draw,fill=white!50, minimum height=10mm,] (Input-\x) at (\x,0) {$x_{\x}$};
        }

    \draw[->, line width=1.5] (Input-1) -- (Stack-1-0);
    \draw[->, line width=1.5] (Input-2) -- (Stack-1-0);
    \draw[->, line width=1.5] (Input-3) -- (Stack-1-0);
    \draw[->, line width=1.5] (Input-4) -- (Stack-1-0);

    \draw[->, line width=1.5] (Input-5) -- (Stack-2-0);
    \draw[->, line width=1.5] (Input-6) -- (Stack-2-0);
    \draw[->, line width=1.5] (Input-7) -- (Stack-2-0);
    \draw[->, line width=1.5] (Input-8) -- (Stack-2-0);

    \draw[->, line width=1.5] (Input-9) -- (Stack-3-0);
    \draw[->, line width=1.5] (Input-10) -- (Stack-3-0);
    \draw[->, line width=1.5] (Input-11) -- (Stack-3-0);
    \draw[->, line width=1.5] (Input-12) -- (Stack-3-0);

    \draw[->, line width=1.5] (Input-13) -- (Stack-4-0);
    \draw[->, line width=1.5] (Input-14) -- (Stack-4-0);
    \draw[->, line width=1.5] (Input-15) -- (Stack-4-0);
    \draw[->, line width=1.5] (Input-16) -- (Stack-4-0);

    % \node[right=of Input-15, align=right] {
    % \textbf{Original Input}
    % };
    % \node[right=of Stack-4, align=right] {
    % \textbf{Stacked Input}
    % };

\end{tikzpicture}%
}\end{figure}
\end{frame}

\begin{frame}{Expanding Receptive Field By Stacking - Results}
\protect\hypertarget{expanding-receptive-field-by-stacking---results}{}
TODO insert result table here
\end{frame}

\begin{frame}{Expanding Receptive Field By Stacking - Conclusions}
\protect\hypertarget{expanding-receptive-field-by-stacking---conclusions}{}
TODO Conclusion
\end{frame}

\begin{frame}{Latent space of stacked WaveNet output - Setup}
\protect\hypertarget{latent-space-of-stacked-wavenet-output---setup}{}
\end{frame}

\begin{frame}{WaveNet as a Language Model - Setup}
\protect\hypertarget{wavenet-as-a-language-model---setup}{}
\end{frame}

\begin{frame}{WaveNet as a Language Model - Results}
\protect\hypertarget{wavenet-as-a-language-model---results}{}
\end{frame}

\begin{frame}{WaveNet as an ASR preprocessor - setup}
\protect\hypertarget{wavenet-as-an-asr-preprocessor---setup}{}
\begin{figure}
    \centering
    \begin{subfigure}[b]{0.5\linewidth}
        \resizebox{\columnwidth}{!}{
            \begin{tikzpicture}
%% WaveNet Structure first


%% 


\node[circle,draw,fill=blue!50,minimum height=5mm,] (Input-15) at (15,0) {};
\node[circle,draw,fill=gray!30,minimum height=5mm, ] (Hidden1-15) at (15,2) {};
\node[circle,draw,fill=gray!30,minimum height=5mm] (Hidden2-15) at (15,4) {};
\node[circle,draw,fill=gray!30,minimum height=5mm] (Hidden3-15) at (15,6) {};
\node[circle,draw,fill=yellow!50,minimum height=5mm] (Output-15) at (15,8) {};



\draw[->, line width=1.5] (15,0.3) -- (15, 1.7);
\draw[->, line width=1.5] (15,2.3) -- (15, 3.7);
\draw[->, line width=1.5] (15,4.3) -- (15, 5.7);
\draw[->, line width=1.5] (15,6.3) -- (15, 7.7);



\foreach \x in {14, 13, ...,0}{
    \node[circle,draw,fill=blue!50,minimum height=5mm] (Input-\x) at (\x,0) {};
    \node[circle,draw,fill=gray!30,minimum height=5mm] (Hidden1-\x) at (\x,2) {};
    \node[circle,draw,fill=gray!30,minimum height=5mm] (Hidden2-\x) at (\x,4) {};
    \node[circle,draw,fill=gray!30,minimum height=5mm] (Hidden3-\x) at (\x,6) {};
    \node[circle,draw,fill=yellow!30,minimum height=5mm] (Output-\x) at (\x,8) {};
}



\foreach \x in {0, 1, ..., 14}{
	\ifthenelse{\intcalcMod{\x}{2}=0}{
		\draw[->, line width=1.5] (Input-\x) -- (Hidden1-\intcalcAdd{\x}{1});
		\draw[->, line width=.5,dashed] (Input-\x) -- (Hidden1-\x);
	}{
		\draw[->, line width=1.5] (Input-\x) -- (Hidden1-\x);
		\draw[->, line width=.5, dashed] (Input-\x) -- (Hidden1-\intcalcAdd{\x}{1});
	}
	\ifthenelse{\intcalcMod{\x}{4}=1}{
   		\draw[->, line width=1.5] (Hidden1-\x) -- (Hidden2-\intcalcAdd{\x}{2});
	}{
	    \ifnodedefined{Hidden2-\intcalcAdd{\x}{2}}{
   		\draw[->, line width=.5, dashed] (Hidden1-\x) -- (Hidden2-\intcalcAdd{\x}{2});
   		}{}
	}
	\ifthenelse{\intcalcMod{\x}{4}=3}{
   		\draw[->, line width=1.5] (Hidden1-\x) -- (Hidden2-\x);
	}{
       	\draw[->, line width=.5,dashed] (Hidden1-\x) -- (Hidden2-\x);
	}
	\ifthenelse{\intcalcMod{\x}{8}=3}{
		\draw[->, line width=1.5] (Hidden2-\x) -- (Hidden3-\intcalcAdd{\x}{4});
	}{
        \ifnodedefined{Hidden3-\intcalcAdd{\x}{4}}{
   		    \draw[->, line width=.5, dashed] (Hidden2-\x) -- (Hidden3-\intcalcAdd{\x}{4});
   		}{}
	}
	\ifthenelse{\intcalcMod{\x}{8}=7}{
   		\draw[->, line width=1.5] (Hidden2-\x) -- (Hidden3-\x);
	}{
   		\draw[->, line width=.5, dashed] (Hidden2-\x) -- (Hidden3-\x);
	}
	\ifthenelse{\intcalcMod{\x}{16}=7}{
		\draw[->, line width=1.5] (Hidden3-\x) -- (Output-\intcalcAdd{\x}{8});
	}{
	\ifnodedefined{Output-\intcalcAdd{\x}{8}}{
   		    \draw[->, line width=.5, dashed] (Hidden3-\x) -- (Output-\intcalcAdd{\x}{8});
   		}{}
	}
	\ifthenelse{\intcalcMod{\x}{16}=15}{
   		\draw[->, line width=1.5] (Hidden3-\x) -- (Output-\x);
	}{
	\draw[->,  line width=.5, dashed] (Hidden3-\x) -- (Output-\x);
	}
}



%% Convolutional Layer

\node[rectangle, draw, rounded corners, above=of Output-15] (Conv-15)  {Conv1D};
\node[rectangle, draw, rounded corners, above=of Output-11] (Conv-11)  {Conv1D};
\node[rectangle, draw, rounded corners, above=of Output-7] (Conv-7)  {Conv1D};
\node[rectangle, draw, rounded corners, above=of Output-3] (Conv-3)  {Conv1D};

\foreach \x in {0, 1, ..., 15}{
\ifthenelse{
    \intcalcMax{\x}{3}=3
}{
    \draw[->, line width=1.5] (Output-\x) -- (Conv-3);
}{
\ifthenelse{
    \intcalcMax{\x}{7}=7
}{
    \draw[->, line width=1.5] (Output-\x) -- (Conv-7);
}{
\ifthenelse{
    \intcalcMax{\x}{11}=11
}{
    \draw[->, line width=1.5] (Output-\x) -- (Conv-11);
}{

    \draw[->, line width=1.5] (Output-\x) -- (Conv-15);
}
}
}
}

\foreach \i in {3,7,11,15}{
\node[rectangle, draw, above=of Conv-\i, minimum height=1cm, minimum width=1cm] (LSTM-\i) {LSTM};
\node[circle, minimum size = 10mm, draw, fill=gray!30, above=of LSTM-\i] (X-hat-\i) {$\hat{x}$};
\draw[->, line width=1.5] (Conv-\i) -- (LSTM-\i);
\draw[->, line width=1.5] (LSTM-\i) -- (X-hat-\i);
\ifthenelse{\i=3}{!}{
\draw[->] (LSTM-\intcalcSub{\i}{4}) -- (LSTM-\i) node[midway, above] {$h_t$};
}
}


\node[above=of X-hat-3] (Text-3) {R};
\node[above=of X-hat-7] (Text-7) {A};
\node[above=of X-hat-11] (Text-11) {I};
\node[above=of X-hat-15] (Text-15) {N};

\draw[->] (X-hat-3) -- (Text-3) {};
\draw[->] (X-hat-7) -- (Text-7) {};
\draw[->] (X-hat-11) -- (Text-11) {};
\draw[->] (X-hat-15) -- (Text-15) {};

% Text and notes
\node[right=of Input-15, align=left] {
\textbf{Input}
};
% \node[right=of Hidden1-15, align=left] {
% \textbf{Hidden1}\\
% Dilation = 1
% };
\node[right=of Hidden2-15, align=left] {
\textbf{Hidden Layers}\\
};
% \node[right=of Hidden3-15, align=left] {
% \textbf{Hidden3}\\
% Dilation = 4
% };
\node[right=of Output-15, align=left] {
\textbf{WaveNet Output}\\
};

\node[right=of Conv-15, align=left] {
\textbf{Convolutional}\\\textbf{Downsampler}
};

\node[right=of LSTM-15, align=left] {
\textbf{LSTM}\\\textbf{ASR Model}
};

\node[right=of X-hat-15, align=left] {
\textbf{Softmax Output}
};

\node[right=of Text-15, align=left] {
\textbf{Decoded Output}
};

\end{tikzpicture}
        }
    %\caption{Setup of the WaveNet-LSTM ASR experiment.}
    %\label{fig:wavenet-asr}
    \end{subfigure}%
    \begin{subfigure}[b]{0.5\linewidth}
        \resizebox{\columnwidth}{!}{
            \begin{tikzpicture}
    % Input
    \foreach \x in {0, 1, ..., 15}{
            \node[circle,draw,fill=blue!50,minimum height=5mm] (Input-\x) at (\x,0) {};
        }

    %% Convolutional Layer

    \node[rectangle, draw, rounded corners, above=of Input-15] (Conv-15)  {Conv1D};
    \node[rectangle, draw, rounded corners, above=of Input-11] (Conv-11)  {Conv1D};
    \node[rectangle, draw, rounded corners, above=of Input-7] (Conv-7)  {Conv1D};
    \node[rectangle, draw, rounded corners, above=of Input-3] (Conv-3)  {Conv1D};

    \foreach \x in {0, 1, ..., 15}{
            \ifthenelse{
                \intcalcMax{\x}{3}=3
            }{
                \draw[->, line width=1.5] (Input-\x) -- (Conv-3);
            }{
                \ifthenelse{
                    \intcalcMax{\x}{7}=7
                }{
                    \draw[->, line width=1.5] (Input-\x) -- (Conv-7);
                }{
                    \ifthenelse{
                        \intcalcMax{\x}{11}=11
                    }{
                        \draw[->, line width=1.5] (Input-\x) -- (Conv-11);
                    }{

                        \draw[->, line width=1.5] (Input-\x) -- (Conv-15);
                    }
                }
            }
        }

    \foreach \i in {3,7,11,15}{
            \node[rectangle, draw, above=of Conv-\i, minimum height=1cm, minimum width=1cm] (LSTM-\i) {LSTM};
            \node[circle, minimum size = 10mm, draw, fill=gray!30, above=of LSTM-\i] (X-hat-\i) {$\hat{x}$};
            \draw[->, line width=1.5] (Conv-\i) -- (LSTM-\i);
            \draw[->, line width=1.5] (LSTM-\i) -- (X-hat-\i);
            \ifthenelse{\i=3}{!}{
                \draw[->] (LSTM-\intcalcSub{\i}{4}) -- (LSTM-\i) node[midway, above] {$h_t$};
            }
        }

    \node[above=of X-hat-3] (Text-3) {R};
    \node[above=of X-hat-7] (Text-7) {A};
    \node[above=of X-hat-11] (Text-11) {I};
    \node[above=of X-hat-15] (Text-15) {N};

    \draw[->] (X-hat-3) -- (Text-3) {};
    \draw[->] (X-hat-7) -- (Text-7) {};
    \draw[->] (X-hat-11) -- (Text-11) {};
    \draw[->] (X-hat-15) -- (Text-15) {};

    % Text and notes
    \node[right=of Input-15, align=left] {
        \textbf{Input}
    };

    \node[right=of Conv-15, align=left] {
        \textbf{Convolutional}\\\textbf{Downsampler}
    };

    \node[right=of LSTM-15, align=left] {
        \textbf{LSTM}\\\textbf{ASR Model}
    };

    \node[right=of X-hat-15, align=left] {
        \textbf{Softmax Output}
    };

    \node[right=of Text-15, align=left] {
        \textbf{Decoded Output}
    };

\end{tikzpicture}
        }
    %\caption{Setup of the LSTM ASR experiment.}
    %\label{fig:lstm-asr}
    \end{subfigure}
    %\caption{
    %Comparison of the WaveNet-LSTM and LSTM ASR experiment setups.
    %}
\end{figure}

\end{frame}

\hypertarget{conclusions}{%
\section{Conclusions}\label{conclusions}}

\end{document}
