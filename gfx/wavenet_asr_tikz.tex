\begin{tikzpicture}
%% WaveNet Structure first


%% 


\node[circle,draw,fill=blue!50,minimum height=5mm,] (Input-15) at (15,0) {};
\node[circle,draw,fill=gray!30,minimum height=5mm, ] (Hidden1-15) at (15,2) {};
\node[circle,draw,fill=gray!30,minimum height=5mm] (Hidden2-15) at (15,4) {};
\node[circle,draw,fill=gray!30,minimum height=5mm] (Hidden3-15) at (15,6) {};
\node[circle,draw,fill=yellow!50,minimum height=5mm] (Output-15) at (15,8) {};



\draw[->, line width=1.5] (15,0.3) -- (15, 1.7);
\draw[->, line width=1.5] (15,2.3) -- (15, 3.7);
\draw[->, line width=1.5] (15,4.3) -- (15, 5.7);
\draw[->, line width=1.5] (15,6.3) -- (15, 7.7);



\foreach \x in {14, 13, ...,0}{
    \node[circle,draw,fill=blue!50,minimum height=5mm] (Input-\x) at (\x,0) {};
    \node[circle,draw,fill=gray!30,minimum height=5mm] (Hidden1-\x) at (\x,2) {};
    \node[circle,draw,fill=gray!30,minimum height=5mm] (Hidden2-\x) at (\x,4) {};
    \node[circle,draw,fill=gray!30,minimum height=5mm] (Hidden3-\x) at (\x,6) {};
    \node[circle,draw,fill=yellow!30,minimum height=5mm] (Output-\x) at (\x,8) {};
}



\foreach \x in {0, 1, ..., 14}{
	\ifthenelse{\intcalcMod{\x}{2}=0}{
		\draw[->, line width=1.5] (Input-\x) -- (Hidden1-\intcalcAdd{\x}{1});
		\draw[->, line width=.5,dashed] (Input-\x) -- (Hidden1-\x);
	}{
		\draw[->, line width=1.5] (Input-\x) -- (Hidden1-\x);
		\draw[->, line width=.5, dashed] (Input-\x) -- (Hidden1-\intcalcAdd{\x}{1});
	}
	\ifthenelse{\intcalcMod{\x}{4}=1}{
   		\draw[->, line width=1.5] (Hidden1-\x) -- (Hidden2-\intcalcAdd{\x}{2});
	}{
	    \ifnodedefined{Hidden2-\intcalcAdd{\x}{2}}{
   		\draw[->, line width=.5, dashed] (Hidden1-\x) -- (Hidden2-\intcalcAdd{\x}{2});
   		}{}
	}
	\ifthenelse{\intcalcMod{\x}{4}=3}{
   		\draw[->, line width=1.5] (Hidden1-\x) -- (Hidden2-\x);
	}{
       	\draw[->, line width=.5,dashed] (Hidden1-\x) -- (Hidden2-\x);
	}
	\ifthenelse{\intcalcMod{\x}{8}=3}{
		\draw[->, line width=1.5] (Hidden2-\x) -- (Hidden3-\intcalcAdd{\x}{4});
	}{
        \ifnodedefined{Hidden3-\intcalcAdd{\x}{4}}{
   		    \draw[->, line width=.5, dashed] (Hidden2-\x) -- (Hidden3-\intcalcAdd{\x}{4});
   		}{}
	}
	\ifthenelse{\intcalcMod{\x}{8}=7}{
   		\draw[->, line width=1.5] (Hidden2-\x) -- (Hidden3-\x);
	}{
   		\draw[->, line width=.5, dashed] (Hidden2-\x) -- (Hidden3-\x);
	}
	\ifthenelse{\intcalcMod{\x}{16}=7}{
		\draw[->, line width=1.5] (Hidden3-\x) -- (Output-\intcalcAdd{\x}{8});
	}{
	\ifnodedefined{Output-\intcalcAdd{\x}{8}}{
   		    \draw[->, line width=.5, dashed] (Hidden3-\x) -- (Output-\intcalcAdd{\x}{8});
   		}{}
	}
	\ifthenelse{\intcalcMod{\x}{16}=15}{
   		\draw[->, line width=1.5] (Hidden3-\x) -- (Output-\x);
	}{
	\draw[->,  line width=.5, dashed] (Hidden3-\x) -- (Output-\x);
	}
}



%% Convolutional Layer

\node[rectangle, draw, rounded corners, above=of Output-15] (Conv-15)  {Conv1D};
\node[rectangle, draw, rounded corners, above=of Output-11] (Conv-11)  {Conv1D};
\node[rectangle, draw, rounded corners, above=of Output-7] (Conv-7)  {Conv1D};
\node[rectangle, draw, rounded corners, above=of Output-3] (Conv-3)  {Conv1D};

\foreach \x in {0, 1, ..., 15}{
\ifthenelse{
    \intcalcMax{\x}{3}=3
}{
    \draw[->, line width=1.5] (Output-\x) -- (Conv-3);
}{
\ifthenelse{
    \intcalcMax{\x}{7}=7
}{
    \draw[->, line width=1.5] (Output-\x) -- (Conv-7);
}{
\ifthenelse{
    \intcalcMax{\x}{11}=11
}{
    \draw[->, line width=1.5] (Output-\x) -- (Conv-11);
}{

    \draw[->, line width=1.5] (Output-\x) -- (Conv-15);
}
}
}
}

\foreach \i in {3,7,11,15}{
\node[rectangle, draw, above=of Conv-\i, minimum height=1cm, minimum width=1cm] (LSTM-\i) {LSTM};
\node[circle, minimum size = 10mm, draw, fill=gray!30, above=of LSTM-\i] (X-hat-\i) {$\hat{x}$};
\draw[->, line width=1.5] (Conv-\i) -- (LSTM-\i);
\draw[->, line width=1.5] (LSTM-\i) -- (X-hat-\i);
\ifthenelse{\i=3}{!}{
\draw[->] (LSTM-\intcalcSub{\i}{4}) -- (LSTM-\i) node[midway, above] {$h_t$};
}
}


\node[above=of X-hat-3] (Text-3) {R};
\node[above=of X-hat-7] (Text-7) {A};
\node[above=of X-hat-11] (Text-11) {I};
\node[above=of X-hat-15] (Text-15) {N};

\draw[->] (X-hat-3) -- (Text-3) {};
\draw[->] (X-hat-7) -- (Text-7) {};
\draw[->] (X-hat-11) -- (Text-11) {};
\draw[->] (X-hat-15) -- (Text-15) {};

% Text and notes
\node[right=of Input-15, align=left] {
\textbf{Input}
};
% \node[right=of Hidden1-15, align=left] {
% \textbf{Hidden1}\\
% Dilation = 1
% };
\node[right=of Hidden2-15, align=left] {
\textbf{Hidden Layers}\\
};
% \node[right=of Hidden3-15, align=left] {
% \textbf{Hidden3}\\
% Dilation = 4
% };
\node[right=of Output-15, align=left] {
\textbf{WaveNet Output}\\
};

\node[right=of Conv-15, align=left] {
\textbf{Convolutional}\\\textbf{Downsampler}
};

\node[right=of LSTM-15, align=left] {
\textbf{LSTM}\\\textbf{ASR Model}
};

\node[right=of X-hat-15, align=left] {
\textbf{Softmax Output}
};

\node[right=of Text-15, align=left] {
\textbf{Decoded Output}
};

\end{tikzpicture}