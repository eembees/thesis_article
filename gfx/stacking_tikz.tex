\begin{tikzpicture}
    \foreach \x in {16, 12, 8, 4} {
            \node[circle, draw, fill=black!20, minimum height=10mm,] (Stack-\intcalcDiv{\x}{4}-0) at (\x, 2.4) {};

            \node[circle, draw, fill=black!20, minimum height=10mm,] (Stack-\intcalcDiv{\x}{4}-1) at (\x, 2.6) {};

            \node[circle, draw, fill=black!20, minimum height=10mm,] (Stack-\intcalcDiv{\x}{4}-2) at (\x, 2.8) {};

            \node[circle, draw, fill=black!20, minimum height=10mm,] (Stack-\intcalcDiv{\x}{4}) at (\x, 3) {$x^*_{\intcalcDiv{\x}{4}}$};
        }


    \foreach \x in {16, 15, ..., 1}{
            \node[circle,draw,fill=white!50, minimum height=10mm,] (Input-\x) at (\x,0) {$x_{\x}$};
        }

    \draw[->, line width=1.5] (Input-1) -- (Stack-1-0);
    \draw[->, line width=1.5] (Input-2) -- (Stack-1-0);
    \draw[->, line width=1.5] (Input-3) -- (Stack-1-0);
    \draw[->, line width=1.5] (Input-4) -- (Stack-1-0);

    \draw[->, line width=1.5] (Input-5) -- (Stack-2-0);
    \draw[->, line width=1.5] (Input-6) -- (Stack-2-0);
    \draw[->, line width=1.5] (Input-7) -- (Stack-2-0);
    \draw[->, line width=1.5] (Input-8) -- (Stack-2-0);

    \draw[->, line width=1.5] (Input-9) -- (Stack-3-0);
    \draw[->, line width=1.5] (Input-10) -- (Stack-3-0);
    \draw[->, line width=1.5] (Input-11) -- (Stack-3-0);
    \draw[->, line width=1.5] (Input-12) -- (Stack-3-0);

    \draw[->, line width=1.5] (Input-13) -- (Stack-4-0);
    \draw[->, line width=1.5] (Input-14) -- (Stack-4-0);
    \draw[->, line width=1.5] (Input-15) -- (Stack-4-0);
    \draw[->, line width=1.5] (Input-16) -- (Stack-4-0);

    % \node[right=of Input-15, align=right] {
    % \textbf{Original Input}
    % };
    % \node[right=of Stack-4, align=right] {
    % \textbf{Stacked Input}
    % };

\end{tikzpicture}%